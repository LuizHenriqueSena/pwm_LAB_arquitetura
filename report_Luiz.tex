\documentclass[11pt]{article}

% Use wide margins, but not quite so wide as fullpage.sty
\marginparwidth 0.5in
\oddsidemargin 0.25in
\evensidemargin 0.25in
\marginparsep 0.25in
\topmargin 0.25in
\textwidth 6in \textheight 8 in
% That's about enough definitions

% multirow allows you to combine rows in columns
\usepackage{multirow}
% tabularx allows manual tweaking of column width
\usepackage{tabularx}
% longtable does better format for tables that span pages
\usepackage{longtable}

\begin{document}
% this is an alternate method of creating a title
%\hfill\vbox{\hbox{Gius, Mark}
%       \hbox{Cpe 456, Section 01}
%       \hbox{Lab 1}
%       \hbox{\today}}\par
%
%\bigskip
%\centerline{\Large\bf Lab 1: Security Audit}\par
%\bigskip
\author{Luiz Henrique Coelho Sena}
\title{A model checker to OpenCl}
\maketitle

\section{Summary}

Ensuring that a reliable tool can check the confidence of graphics applications with stability will help developers to design systems securely. The present research aim to extend a tool that checks the properties of CUDA programs to check properties of applications developed in OpenCL.\linebreak
Due the increase of the demand of graphics processing and the degree of complexity a method to check the systems properties has been necessary. Making computers help us more can be a task that requires a high degree of confidence. A good solution to guarantee the high degree of confidence in developing complex systems is the method model checking.\linebreak
The present research offers an extension to the tool ESBMC-GPU. This tool is a model checker for graphics applications developed in CUDA. The purpose is making it verify ordinary applications in the framework OpenCL and integrate it to the verifier ESBMC-GPU which already verify applications written in CUDA.\linebreak
Model checking is the method used in this research. The expected result is make the model check verifier works with secure and obeying all the properties of a OpenCL verifier. Testing a lot of common programs developed in OpenCL and checking the results will ensure the stability and the confidence of the tool.
The ESBMC-GPU gets a huge potential for developers and it need to cover as many languages as it can. For future researches a lot of extensions is been expected like an extension for OpenGL or upgrades in CUDA operational models or changes in the model checker ESBMC and so other.


\section{Activities}

\begin{itemize}
	\item Initial date: 17/03/2017 -- Final date: 17/04/2017
	\begin{enumerate}
		\item Research proposal for PIBIC {\it (done)}
		\item Create GitHub account {\it (done)}
		\item Create slack accoount {\it (done)}
        \item Create trello account {\it (done)}
        \item Study about github {\it (done)}
	\end{enumerate}

	\item Initial date: 18/04/2017 -- Final date: 17/05/2017
	\begin{enumerate}
		\item Write the kanban in trello {\it (done)}
		\item Make a lay summary {\it (done)}
	\end{enumerate}
	\item Initial date: 18/05/2017 -- Final date: 05/06/2017
    \begin{enumerate}
    	\item Count the number of functions that ESBMC-GPU support of CUDA's APIs{\it (done)}
		\item Count the number of missing functions that ESBMC-GPU doesn't support of CUDA's APIs {\it (done)}
        \item Calculate the percentage that ESBMC-GPU support of each API {\it (done)}
      \end{enumerate}
    \item Initial Date: 05/06/2017 -- Final Date: 12/06/2017
	\begin{enumerate}
		\item Lay Summary corrections {\it (failed)}
		\item Generate the graphic points that show the comparative time and tests of ESBMC-GPU and others{\it (done)}
        \item Create a repport in latex{\it (done)}
        \item Decide between OpenCl and Cuda for the present research{\it (to do)}
	\end{enumerate}
		\item Initial Date: 12/06/2017 -- Final Date: 18/06/2017
			\begin{enumerate}
				\item ESBMC-GPU performance comparative graphics corrections {\it (done)}
				\item Decide between OpenCl and Cuda for the present research{\it (done)}
				\item Lay Summary corrections {\it (doing)}
			\end{enumerate}
			\item Initial Date: 18/06/2017 -- Final Date: 25/06/2017
			\begin{enumerate}
				\item Decided between OpenCl and CUDA for the present research: CUDA {\it (done)}
				\item Lay Summary corrections {\it (failed)}
			\end {enumerate}
			\item Initial Date: 25/06/2017 -- Final Date: 02/07/2017
			\begin{enumerate}
				\item Study the current UML diagram of CUDA OM {\it (to do)}
				\item Study the current regression of CUDA OM {\it (to do)}
				\item Lay Summary corrections {\it (to do)}
			\end {enumerate}

\end{itemize}

\end{document}
